\documentclass{beamer}

\usepackage[T1]{fontenc}
\usepackage[brazil]{babel}
\usepackage[utf8]{inputenc}

% Presentation strategy:
% Show previous presentation persona pains - OK
% Present research questions - OK
% Show small research keywords used
% Describe lack of AI-assisted DT development
% Highlight potential benefits of AI integration (e.g.: reduce costs, democratize access, etc)
% Present abstract
% Present research approach (MCP-based AI tools)
% - Explain why MCP
% - Explain how MCP can be used to create AI tools for DT development
% -- Explain MCP architecture
% -- Explain how MCP can be used to create AI agents and workflows
% - Present example tools to be developed
% -- CRD generator from ontology
% -- Tool configuration automation
% -- Cross-platform integration


% Choose the Inf theme
\usetheme{Inf}

% Define the title with \title[short title]{long title}
% Short title is optional
\title[Research study plan]
      {Research study plan}

% Optional subtitle
\subtitle{Research study plan for AI-assisted Digital Twin development}

\date{August 2025}

% Author information
\author{Henrique Krausburg Correa}
\institute{Institute of Informatics --- UFRGS}

\begin{document}

% Command to create title page
\InfTitlePage

\begin{frame}
  \frametitle{Agenda}
  \tableofcontents
\end{frame}

\section{Research question prospect}

\begin{frame}[plain]
 \sectionpage
\end{frame}

\frame{
    \frametitle{Case study}

    This research uses KTWIN \cite{Wermann_Wickboldt_2025} — a serverless Kubernetes-based Digital Twin platform — as a representative case study.
}

\frame{
    \frametitle{Roles}

    KTWIN workflow is based on three primary personas:

    \begin{itemize}
        \item DT Domain Expert
        \begin{itemize}
            \item Defines the Digital Twin model and its requirements.
            \item Responsible for the strategic direction and governance of the DT ecosystem.
            \item Verifies implementation.
        \end{itemize}
        \item DT Developer
        \begin{itemize}
            \item Implements the Digital Twin model using the KTWIN platform.
        \end{itemize}
        \item DT Operator
        \begin{itemize}
            \item Manages the deployment and operation of the Digital Twin in production environments.
            \item Creates CRD (Custom Resource Definition) files to model DT use case.
            \item Monitors system behavior through KTWIN observability solution (Grafana).
        \end{itemize}
    \end{itemize}
}

\frame{
    \frametitle{Insights}

    The following insights were gathered based on persona's tasks described in \cite{Wermann_Wickboldt_2025}:

    \begin{itemize}
        \item DT Domain expert is the most overloaded persona
        \begin{itemize}
            \item Responsible for both strategic modeling tasks and operational configuration duties.
            \item Faces challenges in managing the complexity of the Digital Twin ecosystem.
        \end{itemize}
        \item DT Developer is the “weakest link”
        \begin{itemize}
            \item Has few responsibilities compared to other personas.
        \end{itemize}
        \item DT Operator is often underestimated
        \begin{itemize}
            \item Could absorb DT domain expert / developer duties if empowered with an AI assistant.
        \end{itemize}
    \end{itemize}
}

\frame{
    \frametitle{Research questions}

    \begin{itemize}
        \item \textbf{RQ1:} How has artificial intelligence been used in Digital Twin development processes so far, and what gaps exist in current approaches?
        \item \textbf{RQ2:} How can artificial intelligence improve Digital Twin modeling, development, deployment, and operability through automated assistance and knowledge augmentation?
        \item \textbf{RQ3:} Can artificial intelligence contribute to solving interoperability issues in Digital Twin development through standardized integration frameworks and automated tool orchestration?
        \end{itemize}
}

\section{Literature review}

\begin{frame}[plain]
 \sectionpage
\end{frame}


\frame{
    \frametitle{AI in Digital Twin development}

    To address \textbf{RQ1}, a preliminary literature review was conducted using Google Scholar with the following keywords:
    \begin{itemize}
        \item AI-Powered Digital Twin
        \item Digital Twin enabling technologies
        \item AI Digital Twin development
    \end{itemize}
}

\frame{
    \frametitle{Key findings}

    The top 3 relevant papers (Google Scholar rank) were analyzed. Key findings include:
    \begin{itemize}
        \item Enabling technologies papers highlight the high complexity of Digital Twin development due to heterogeneous technologies and the vast number of required tools.
        \begin{itemize}
            \item \cite{Fuller_Fan_Day_Barlow_2020}, \cite{Hu_Zhang_Deng_Liu_Tan_2021}, \cite{Qi_Tao_Hu_Anwer_Liu_Wei_Wang_Nee_2021}.
        \end{itemize}
        \item Most AI applications are focused on enhancing production environments as part of the final Digital Twin solution.
         \begin{itemize}
            \item \cite{Alnaser_Maxi_Elmousalami_2024}, \cite{Sarp_Kuzlu_Jovanovic_Polat_Guler_2024}, \cite{Avanzato_Beritelli_Lombardo_Ricci_2024}
        \end{itemize}
        \item There is a lack of AI-powered tools or copilots designed to assist developers during the Digital Twin development process itself.
    \end{itemize}
}

\section{Abstract draft}

\begin{frame}[plain]
 \sectionpage
\end{frame}

\frame{
    \frametitle{Context}
    Digital Twins have emerged as transformative technology for creating virtual replicas of physical systems, enabling real-time monitoring, simulation, and optimization across various industries.

    \begin{block}{}
        However, their development remains complex and resource-intensive, requiring significant expertise across multiple technological domains and creating substantial barriers to adoption, particularly for small and medium-sized enterprises.
    \end{block}
    
    Although recent advances in Artificial Intelligence, particularly Large Language Models (LLMs) and AI-assisted software development tools, the application of AI to Digital Twin development processes remains largely unexplored.
}

\frame{
    \frametitle{Proposal}
    This research aims to develop generalizable AI-assisted methodologies to enhance productivity throughout the Digital Twin development process, addressing gaps in current development methodologies. 

    \begin{block}{}
    The proposed approach leverages the Model Context Protocol (MCP) specification to create specialized AI tools and connectors that automate Digital Twin development tasks, facilitate cross-platform integration, and bridge knowledge gaps between heterogeneous toolchains.
    \end{block}

    KTWIN serves as a practical testbed for implementing and evaluating these AI tools, enabling demonstration of their effectiveness in automating key development tasks, improving workflow efficiency, and supporting interoperability within a real-world Digital Twin platform.
}

\frame{
    \frametitle{Cointributions}
    Expected contributions include:
    \begin{itemize}
        \item A comprehensive analysis of current AI applications in Digital Twin development.
        \item A novel MCP-based framework for AI-assisted development processes.
        \item Practical tools demonstrating measurable productivity improvements.
        % \item Validation through comparative development studies, tool integration metrics, and case study analysis using smart city scenarios.
        \item Validation ?????
    \end{itemize}
}

\section{Enabling technologies}

\begin{frame}[plain]
 \sectionpage
\end{frame}

\frame{
    \frametitle{Model Context Protocol (MCP)}

    MCP is an open specification for context-aware model integration and orchestration in Digital Twin ecosystems.

    \begin{itemize}
        \item Defines standardized interfaces for exchanging model context, metadata, and operational parameters.
        \item Enables interoperability between heterogeneous tools and platforms.
        \item Facilitates automated workflows and AI agent integration.
        \item Serves as the foundation for building AI-assisted connectors and development tools.
    \end{itemize}
}

\frame{
    \frametitle{GitHub Copilot}

    GitHub Copilot is an AI-powered coding assistant developed by GitHub and OpenAI.

    \begin{itemize}
        \item Provides real-time code suggestions and autocompletions in popular IDEs.
        \item Trained on a large corpus of public code, enabling context-aware recommendations.
        \item Accelerates software development and reduces manual coding effort.
        \item Inspires the concept of AI copilots for Digital Twin development tasks.
    \end{itemize}
}

\frame{
    \frametitle{KubeFlow}

    KubeFlow is an open-source platform for deploying, managing, and scaling machine learning workflows on Kubernetes.

    \begin{itemize}
        \item Provides a suite of tools for end-to-end ML lifecycle management, including training, serving, and monitoring.
        \item Integrates seamlessly with Kubernetes, enabling scalable and reproducible ML pipelines.
        \item Supports automation and orchestration of complex workflows, making it suitable for Digital Twin scenarios.
        \item Offers extensibility for integrating AI agents and custom connectors in Digital Twin development.
    \end{itemize}
}

\frame{
    \frametitle{LangChain}

    LangChain is an open-source framework for developing applications powered by Large Language Models (LLMs).

    \begin{itemize}
        \item Enables chaining of LLMs with external tools, APIs, and data sources.
        \item Supports agent-based workflows and context-aware automation.
        \item Facilitates rapid prototyping of AI assistants and copilots.
        \item Provides building blocks for integrating LLMs into Digital Twin development pipelines.
    \end{itemize}
}

\section{Next steps}

\begin{frame}[plain]
 \sectionpage
\end{frame}

\frame{
    \frametitle{TODO}
    \begin{enumerate}
        \item Define KTWIN task to be solved by AI
        \item Define Methodology
        \item Define Validation
    \end{enumerate}

}

\section*{}

\begin{frame}
    \frametitle{Thanks!}
    \InfContacts
\end{frame}

\begin{frame}[allowframebreaks]
    \frametitle{References}
    \bibliographystyle{abntex2-alf}
    \bibliography{presentation}
\end{frame}

\end{document}
