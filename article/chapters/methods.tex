\section{Materiais e Métodos}
\label{sec:methods}

\subsection{Descrição do Dataset}

% Fonte dos dados (link para Kaggle)
% Características gerais (42.675 instâncias, 17 atributos)
% Tabela com descrição dos atributos
% Variável alvo (purchased_last_month)

O presente estudo utiliza o \textit{Amazon Products Sales Dataset 42K+ Items - 2025}~\cite{sherazi2025},
    disponível publicamente no Kaggle\footnote{\url{https://www.kaggle.com/datasets/ikramshah512/amazon-products-sales-dataset-42k-items-2025}}.
Este dataset contém informações detalhadas sobre 42.675 produtos da categoria eletrônicos
    comercializados na plataforma Amazon, coletados em setembro de 2025.

\subsubsection{Características Gerais}

O dataset original disponibiliza duas versões:
\textit{cleaned} (dados pré-processados) e
\textit{uncleaned} (dados brutos).
Para este trabalho, utilizamos a versão \textbf{cleaned} como referência inicial,
    realizando posteriormente etapas adicionais de pré-processamento
    conforme descrito na Seção~\ref{sec:preprocessing}.

O dataset compreende \textbf{42.675 instâncias} com \textbf{17 atributos},
    categorizados em três grupos funcionais:
\textit{características de produto} (ratings, reviews, preço),
\textit{estratégias de marketing} (patrocínio, cupons, selos) e
\textit{metadados auxiliares} (URLs, datas, imagens).

\subsubsection{Atributos do Dataset}

A Tabela~\ref{tab:dataset-features} descreve os 17 atributos presentes no dataset,
    indicando tipo de dado, categoria funcional e relevância para o estudo.

\begin{table}[htbp]
\centering
\caption{Descrição dos atributos do Amazon Products Sales Dataset}
\label{tab:dataset-features}
\small
\begin{tabular}{@{}llp{4.5cm}l@{}}
    \toprule
    \textbf{Atributo} & \textbf{Tipo} & \textbf{Descrição} & \textbf{Categoria} \\
    \midrule
    \texttt{product\_title} & Textual & Nome completo do produto & Metadado \\
    \addlinespace
    \texttt{product\_rating} & Numérico & Avaliação média (0-5) & Produto \\
    \addlinespace
    \texttt{total\_reviews} & Inteiro & Número total de avaliações & Produto \\
    \addlinespace
    \texttt{purchased\_last\_month} & Inteiro & \textbf{Unidades vendidas no mês} & \textbf{Produto} \\
    & & \textbf{anterior (variável alvo)} & \\
    \addlinespace
    \texttt{discounted\_price} & Numérico & Preço atual com desconto (\$) & Marketing \\
    \addlinespace
    \texttt{original\_price} & Numérico & Preço original sem desconto (\$) & Produto \\
    \addlinespace
    \texttt{discount\_percentage} & Numérico & Percentual de desconto aplicado & Marketing \\
    \addlinespace
    \texttt{is\_best\_seller} & Booleano & Selo "Best Seller" da Amazon & Marketing \\
    \addlinespace
    \texttt{is\_sponsored} & Booleano & Produto patrocinado (paid placement) & Marketing \\
    \addlinespace
    \texttt{has\_coupon} & Booleano & Disponibilidade de cupom de desconto & Marketing \\
    \addlinespace
    \texttt{buy\_box\_availability} & Booleano & Disponibilidade do botão "Add to Cart" & Marketing \\
    \addlinespace
    \texttt{delivery\_date} & Temporal & Data estimada de entrega & Metadado \\
    \addlinespace
    \texttt{sustainability\_tags} & Textual & Tags de sustentabilidade (eco-friendly) & Marketing \\
    \addlinespace
    \texttt{product\_image\_url} & URL & Link da imagem do produto & Metadado \\
    \addlinespace
    \texttt{product\_page\_url} & URL & Link da página oficial do produto & Metadado \\
    \addlinespace
    \texttt{data\_collected\_at} & Temporal & Data de coleta dos dados & Metadado \\
    \addlinespace
    \texttt{product\_category} & Categórico & Categoria atribuída com base no título & Produto \\
    \bottomrule
\end{tabular}
\end{table}

\subsubsection{Variável Alvo}

A variável alvo deste estudo é \texttt{purchased\_last\_month},
    que representa o \textbf{número de unidades vendidas no mês anterior à coleta dos dados}.
Esta variável define uma tarefa de \textbf{regressão},
    onde o objetivo é prever o volume de vendas mensais
    com base em características de produto e estratégias de marketing.

A escolha desta variável como target fundamenta-se em três aspectos:

\begin{enumerate}
    \item \textbf{Alinhamento com objetivo de pesquisa:}
        Permite investigar correlações entre características observáveis no momento da coleta
        (ratings, preço, selos) e desempenho de vendas no período imediatamente anterior,
        viabilizando análise transversal do impacto produto vs. marketing.

    \item \textbf{Relevância prática:}
        Vendas mensais constituem métrica direta de desempenho comercial em e-commerce,
        sendo indicador-chave para decisões de estoque, pricing e marketing;
\end{enumerate}

\subsubsection{Categorização Conceitual das Features}

Para responder à questão de pesquisa central deste trabalho,
    categorizamos os 17 atributos em três grupos funcionais:

\begin{itemize}
    \item \textbf{Características de Produto (5 features):}
        Atributos intrínsecos que refletem qualidade percebida e valor do produto:
        \texttt{product\_rating},
        \texttt{total\_reviews},
        \texttt{original\_price},
        \texttt{product\_category},
        \texttt{purchased\_last\_month} (alvo).
        
        \item \textbf{Estratégias de Marketing (7 features):}
        Táticas promocionais e selos de plataforma que aumentam visibilidade:
        \texttt{discounted\_price},
        \texttt{discount\_percentage},
        \texttt{is\_best\_seller},
        \texttt{is\_sponsored},
        \texttt{has\_coupon},
        \texttt{buy\_box\_availability},
        \texttt{sustainability\_tags}.
    
    \item \textbf{Metadados (6 features):}
        Atributos textuais, temporais ou identificadores únicos
        sem poder preditivo direto para análise transversal:
        \texttt{product\_title},
        \texttt{delivery\_date},
        \texttt{product\_image\_url},
        \texttt{product\_page\_url},
        \texttt{data\_collected\_at}.
\end{itemize}

\textbf{Nota sobre escopo analítico:}
Reconhecemos que \texttt{product\_title}, \texttt{delivery\_date} e \texttt{sustainability\_tags}
    possuem potencial preditivo significativo:
\texttt{product\_title} contém informações sobre marca, especificações técnicas e linha de produto
    (extraíveis via NLP);
\texttt{delivery\_date} permite derivar tempo de entrega
    (indicador de disponibilidade em estoque e urgência logística);
\texttt{sustainability\_tags} pode ser codificado como presença/ausência de certificações eco-friendly.
Entretanto, a exploração dessas features requer técnicas de feature engineering
    (NLP, engenharia de features temporais)
    que fogem do escopo central deste trabalho:
    quantificar a importância relativa de características de produto e estratégias de marketing
    diretamente observáveis via modelos tree-based.
Essas features são identificadas como oportunidades para extensões futuras na Seção~\ref{sec:conclusion}.

Esta categorização será utilizada na Seção~\ref{sec:interpretability}
    para agregação de SHAP values por grupo conceitual,
    quantificando a importância relativa de produto versus marketing
    para predição de \texttt{purchased\_last\_month}.

\subsection{Análise Exploratória}

% Distribuição da variável alvo
% Análise de correlações
% Identificação de outliers
% Valores faltantes
% Análise de atributos categóricos vs. numéricos
% Principais padrões identificados

\subsection{Pré-processamento dos Dados}
\label{sec:preprocessing}

% 5.3.1 Feature Engineering

% Tratamento de variáveis categóricas (encoding)
% Criação de features derivadas (se aplicável)

% 5.3.2 Data Cleaning

% Tratamento de valores faltantes
% Remoção/tratamento de outliers
% Normalização/padronização de atributos numéricos

% 5.3.3 Feature Selection

% Remoção de features irrelevantes
%"Atributos textuais (\texttt{product_title}), temporais (\texttt{delivery_date}),
% e identificadores únicos (URLs) foram excluídos da análise por apresentarem
% alta cardinalidade ou natureza não preditiva para análise transversal."
% Análise de importância de features (se aplicável)

\subsection{Desenvolvimento dos Modelos}

% 5.4.1 Algorithms Selection

% Justificativa dos algoritmos escolhidos
% Diversidade de viés indutivo

% 5.4.2 Evaluation Strategy

% Métricas de avaliação (MSE, RMSE, MAE, R²)
% Métrica principal para seleção de modelos
% Estratégia de divisão de dados (holdout, k-fold CV)
% Reprodutibilidade (random_state, seeds)

% 5.4.3 Spot-Checking (se aplicável)

% Avaliação inicial dos algoritmos
% Seleção de candidatos para otimização

% 5.4.4 Hyperparameter Optimization

% Estratégia adotada (Grid Search, Random Search, etc.)
% Espaço de busca dos hiperparâmetros
% Validação cruzada para otimização

\subsection{Técnicas de Interpretabilidade}

% - SHAP values: fundamentação teórica (Shapley values)
% - TreeSHAP: implementação para tree ensembles
% - Agregação por categoria conceitual (produto vs. marketing)

