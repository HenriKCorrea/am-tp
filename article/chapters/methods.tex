\section{Materiais e Métodos}
\label{sec:methods}

\subsection{Descrição do Dataset}

% Fonte dos dados (link para Kaggle)
% Características gerais (42.675 instâncias, 17 atributos)
% Tabela com descrição dos atributos
% Variável alvo (purchased_last_month)

\subsection{Análise Exploratória}

% Distribuição da variável alvo
% Análise de correlações
% Identificação de outliers
% Valores faltantes
% Análise de atributos categóricos vs. numéricos
% Principais padrões identificados

\subsection{Pré-processamento dos Dados}

% 5.3.1 Feature Engineering

% Tratamento de variáveis categóricas (encoding)
% Criação de features derivadas (se aplicável)

% 5.3.2 Data Cleaning

% Tratamento de valores faltantes
% Remoção/tratamento de outliers
% Normalização/padronização de atributos numéricos

% 5.3.3 Feature Selection

% Remoção de features irrelevantes
%"Atributos textuais (\texttt{product_title}), temporais (\texttt{delivery_date}),
% e identificadores únicos (URLs) foram excluídos da análise por apresentarem
% alta cardinalidade ou natureza não preditiva para análise transversal."
% Análise de importância de features (se aplicável)

\subsection{Desenvolvimento dos Modelos}

% 5.4.1 Algorithms Selection

% Justificativa dos algoritmos escolhidos
% Diversidade de viés indutivo

% 5.4.2 Evaluation Strategy

% Métricas de avaliação (MSE, RMSE, MAE, R²)
% Métrica principal para seleção de modelos
% Estratégia de divisão de dados (holdout, k-fold CV)
% Reprodutibilidade (random_state, seeds)

% 5.4.3 Spot-Checking (se aplicável)

% Avaliação inicial dos algoritmos
% Seleção de candidatos para otimização

% 5.4.4 Hyperparameter Optimization

% Estratégia adotada (Grid Search, Random Search, etc.)
% Espaço de busca dos hiperparâmetros
% Validação cruzada para otimização

\subsection{Técnicas de Interpretabilidade}

% - SHAP values: fundamentação teórica (Shapley values)
% - TreeSHAP: implementação para tree ensembles
% - Agregação por categoria conceitual (produto vs. marketing)

