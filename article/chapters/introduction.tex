\section{Introdução}
\label{sec:introduction}

\subsection{Contexto e Motivação}

O crescimento exponencial do comércio eletrônico nas últimas décadas transformou radicalmente
    a dinâmica de vendas no varejo global.
Marketplaces digitais como Amazon, Mercado Livre e AliExpress concentram milhões de produtos
    competindo pela atenção de consumidores,
    onde características de produto
    (e.g., avaliações, preço, qualidade percebida) e
    estratégias de marketing
    (e.g., descontos, cupons, garantias de vendedor)
    constituem fatores multidimensionais que influenciam decisões de compra~\cite{zhao2020}.
Entretanto, estudos que integram análise sistemática dessas categorias
    (produto versus marketing)
    permanecem escassos~\cite{zhao2020},
    especialmente em abordagens que quantifiquem sua importância relativa
    para performance de vendas.

Neste cenário hipercompetitivo, uma questão fundamental emerge para vendedores e plataformas:
    \textbf{investir em melhorias intrínsecas do produto}
    (e.g., qualidade refletida em avaliações, ajuste de preço)
    gera maior retorno em vendas do que
    \textbf{alocar recursos em visibilidade e promoções}
    (e.g., patrocínio de posicionamento, cupons de desconto)?
A efetividade de modelos preditivos de vendas em e-commerce
    depende fortemente da qualidade e organização das features selecionadas~\cite{diwandari2025}.

Técnicas de interpretabilidade como SHAP (SHapley Additive exPlanations)~\cite{NIPS2017_8a20a862}
    permitem quantificar a contribuição marginal de cada feature para predições de modelos tree-based,
    fornecendo medidas rigorosas de importância fundamentadas em teoria dos jogos (Shapley values).
Diferentemente de feature importance nativa de Random Forests ou XGBoost
    (baseada em ganho de impureza ou gain),
    SHAP values oferecem \textbf{interpretabilidade consistente} entre diferentes modelos,
    permitindo comparação objetiva de importância de features
    e análise de interações não-lineares entre variáveis~\cite{lundberg2020}.

\subsection{Objetivo do Trabalho}

Este trabalho tem como objetivo central responder à seguinte questão de pesquisa:

\begin{quote}
\textit{\textbf{Quais fatores são mais determinantes para o volume de vendas mensais de produtos em marketplace digital:
    características intrínsecas do produto (preço, avaliações, ratings) ou
    estratégias de marketing (patrocínio, cupons, selos de plataforma)?}}
\end{quote}

Para investigar esta questão, desenvolvemos uma metodologia de
    \textbf{análise interpretativa transversal} utilizando o
    Amazon Products Sales Dataset 
    (42.675 produtos eletrônicos, 17 features, setembro/2025)~\cite{sherazi2025},
    aplicando técnicas de interpretabilidade ML baseadas em SHAP values.
Especificamente, os objetivos são:

\begin{enumerate}
    \item \textbf{Categorizar features} em dois grupos conceituais:
    \begin{itemize}
        \item \textit{Produto}: \texttt{product\_rating}, \texttt{total\_reviews},
              \texttt{original\_price}
        \item \textit{Marketing}: \texttt{is\_best\_seller}, \texttt{is\_sponsored}, 
              \texttt{has\_coupon}, \texttt{sustainability\_tags},
              \texttt{discounted\_price}, \texttt{discount\_percentage}
    \end{itemize}
    
    \item \textbf{Treinar e avaliar modelos tree-based} (XGBoost, Random Forest, LightGBM)
        para predição de \texttt{purchased\_last\_month} (vendas mensais),
        utilizando validação cruzada estratificada e métricas de regressão (MAE, RMSE, $R^2$);
    
    \item \textbf{Quantificar importância relativa} de cada categoria
        (produto vs. marketing)
        via agregação de SHAP values absolutos médios por grupo de features,
        fornecendo medida objetiva de contribuição marginal para vendas
        em escala percentual;
    
    \item \textbf{Identificar features individuais mais determinantes}
        através de análise de SHAP values globais (beeswarm plots),
        revelando quais atributos específicos dentro de cada categoria
        têm maior poder preditivo para \texttt{purchased\_last\_month};
\end{enumerate}

\subsection{Contribuições}

As principais contribuições deste trabalho são:

\begin{enumerate}
    \item \textbf{Quantificação empírica da dicotomia produto vs. marketing:}
        Agregação de SHAP values por categoria conceitual para medir objetivamente a
        importância relativa de características de produto versus estratégias de marketing
        em vendas de marketplace digital, fornecendo resposta quantitativa
        (em escala percentual de impacto)
        ao trade-off de alocação de recursos enfrentado por vendedores.
    
    \item \textbf{Metodologia de interpretabilidade unificada:}
        Aplicação de SHAP values para comparação rigorosa de importância de features
        entre XGBoost, Random Forest e LightGBM, garantindo interpretabilidade consistente
        através de métrica fundamentada em teoria dos jogos (Shapley values)
    
    \item \textbf{Dataset contemporâneo de eletrônicos Amazon:}
        Análise de 42.675 produtos eletrônicos com 17 features coletados em setembro/2025,
        incluindo atributos específicos de marketplace digital moderno
        (\texttt{is\_sponsored}, \texttt{has\_coupon}, \texttt{sustainability\_tags})
\end{enumerate}

% \subsection{Organização do Artigo}

% O restante deste artigo está organizado da seguinte forma:
% A Seção~\ref{sec:related-work} apresenta revisão de trabalhos relacionados
%     sobre previsão de vendas em e-commerce e análise de importância de features,
%     posicionando este estudo em relação à literatura existente.
% A Seção~\ref{sec:methodology} descreve detalhadamente a metodologia empregada,
%     incluindo análise exploratória dos dados (EDA),
%     pré-processamento,
%     estratégia de treinamento e validação de modelos,
%     e técnicas de interpretabilidade via SHAP.
% A Seção~\ref{sec:results} apresenta os resultados experimentais,
%     incluindo desempenho preditivo dos modelos,
%     análise de importância de features agregada por categoria,
%     e investigação de interações produto-marketing.
% A Seção~\ref{sec:discussion} discute criticamente os achados,
%     limitações do estudo e implicações práticas.
% Finalmente, a Seção~\ref{sec:conclusion} sumariza as principais conclusões
%     e sugere direções para trabalhos futuros.
