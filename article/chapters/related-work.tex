\section{Trabalhos Relacionados}
\label{sec:related-work}

Para contextualizar este trabalho, conduzimos uma busca sistemática simplificada na literatura recente sobre
    previsão de vendas em e-commerce utilizando Aprendizado de Máquina, com foco em
    análise de importância de features.

\subsection{Metodologia de Busca}

A seleção de trabalhos relacionados seguiu a seguinte estratégia:

\textbf{Query de busca:}\\
\texttt{("sales~prediction"~OR "sales~forecasting")~AND}\\
\texttt{("e-commerce"~OR "online~retail")~AND}\\
\texttt{("machine~learning"~OR "regression")~AND}\\
\texttt{("feature~importance"~OR "product~characteristics"~OR "marketing~strategies")}

\textbf{Bibliotecas digitais consultadas:}
\begin{itemize}
    \item IEEE Xplore (\url{https://ieeexplore.ieee.org/})
    \item ScienceDirect (\url{https://www.sciencedirect.com/})
    \item SpringerLink (\url{https://link.springer.com/})
\end{itemize}

\textbf{Critérios de inclusão:}
\begin{itemize}
    \item Publicações entre 2019-2025
    \item Abordagem de análise de vendas em contexto de e-commerce ou varejo online
    \item Uso de técnicas de Machine Learning (preferencialmente tree-based ou 
          com análise de feature importance)
    \item Texto completo disponível via acesso institucional
    \item \textbf{Pelo menos um dos seguintes:}
    \begin{itemize}
        \item Análise explícita de importância de features de produto ou marketing
        \item Framework conceitual de categorização de features (e.g., produto vs. marketing)
        \item Uso de técnicas de interpretabilidade (SHAP, feature importance, etc.)
    \end{itemize}
\end{itemize}

\textbf{Critérios de exclusão:}
\begin{itemize}
    \item Trabalhos puramente metodológicos sem aplicação em contexto de vendas/e-commerce
    \item Trabalhos de classificação (e.g., churn, categoria de produto) quando 
          o foco for exclusivamente categórico sem análise de volume/valor
    \item Trabalhos sem qualquer análise de importância ou interpretabilidade de features
    \item Estudos focados exclusivamente em comportamento de navegação/clickstream 
          sem conexão com vendas realizadas
\end{itemize}


\textbf{Estratégia de seleção:}
Para cada biblioteca digital, os resultados foram ordenados por relevância.
Dada a \textbf{escassez de trabalhos com abordagem puramente transversal} focada em
    características de produto versus marketing, adotamos uma estratégia pragmática de seleção:

\begin{enumerate}
    \item Priorizamos trabalhos que analisam features de produto e/ou marketing, 
          mesmo que utilizem abordagens temporais (desde que não sejam 
          exclusivamente baseadas em padrões históricos agregados)
    \item Selecionamos trabalhos que demonstrem análise de feature importance 
          ou interpretabilidade, independentemente da metodologia temporal/transversal
    \item Para cada biblioteca, buscamos identificar \textbf{1-2 trabalhos} que 
          melhor se alinham aos critérios acima
    \item Artigos com títulos e abstracts promissores tiveram introdução e 
          metodologia analisadas para validação de alinhamento antes da inclusão
\end{enumerate}

A busca resultou em \textbf{3-4 trabalhos} que,
    embora não sejam perfeitamente alinhados ao nosso posicionamento transversal,
    fornecem base conceitual e metodológica relevante.
A escassez identificada reforça a contribuição deste trabalho ao analisar especificamente a dicotomia
    produto versus marketing em dados de marketplace digital.

\subsection{Análise dos Trabalhos}

\textbf{Sharma et al. (2019) - Analysis of Book Sales Prediction at Amazon Marketplace in India: A Machine Learning Approach}

\cite{sharma2019} investigam previsão de vendas de livros no marketplace Amazon India utilizando múltiplas técnicas de modelagem:
    regressão múltipla,
    Decision Tree (C5.0) e
    Artificial Neural Networks (ANN).
O estudo utiliza 1.408 livros com 16.000 reviews coletados em dezembro/2017, analisando
    fatores de produto (
            \textit{price},
            \textit{average ratings},
            \textit{review volume}, 
            \textit{number of pages}) e
    marketing (
        \textit{discount rate},
        \textit{discount amount},
        \textit{free delivery}).
Análise de sentimento (Sentiment Analysis API e SentiStrength) extrai três variáveis adicionais:
    \textit{overall polarity},
    \textit{positive sentiment strength} e
    \textit{negative sentiment strength}.
Seis efeitos de interação são incluídos:
    Discount Rate × Review Volume,
    Discount Rate × Average Ratings,
    Review Volume × Positive/Negative Sentiment,
    Discount Rate × Positive/Negative Sentiment.

\textbf{Principais resultados:} 
\textbf{(1) Feature importance:} Todos os três modelos confirmam \textit{review volume} como preditor mais importante de vendas.
Decision Tree reportou review volume com maior importância, seguido por interações com sentimentos.
ANN (ensemble de 10 MLPs) atribuiu importância de
    0.17 para review volume,
    0.17 para Review Volume × Positive Sentiment e
    0.16 para Discount Rate × Review Volume. 
\textbf{(2) Achados contraditórios sobre marketing:}
    \textit{Discount rate} e \textit{discount amount} apresentaram efeito mínimo ou insignificante nas vendas
    ($p>0.05$ na regressão, importância 0.01-0.06 em DT/ANN),
    contradizendo teoria de transaction utility. \textit{Average ratings} também não foi significativo em nenhum modelo.
\textbf{(3) Sentimento:} Regressão identificou positive/negative sentiment como
    significativos ($p<0.01$, coeficientes 0.19/-0.27),
    mas ANN reportou importância zero (0.0) para ambos como features isoladas, sugerindo que
    sentimento afeta vendas apenas via interações com review volume.
\textbf{(4) Interações:} Review Volume × Negative Sentiment e Discount Rate × Review Volume
    foram significativos em todos os modelos,
    demonstrando que descontos são efetivos quando review volume é alto.
\textbf{(5) Performance:} ANN ensemble alcançou R² de 0.708 (70.8\% accuracy),
    superando regressão (R²=0.52) e Decision Tree (66.9\% accuracy).

\textbf{Relevância para este trabalho:}
Este estudo fornece evidência empírica direta sobre importância
    relativa produto vs. marketing em \textbf{marketplace Amazon} (contexto idêntico ao nosso dataset),
    utilizando feature importance explícita em modelos tree-based (Decision Tree C5.0).
O achado central de que \textit{review volume} (característica de produto refletindo popularidade/qualidade percebida)
    domina sobre \textit{discount rate/amount} (estratégias de marketing)
    alinha-se com nossa hipótese de investigação sobre dicotomia produto vs. marketing.
A descoberta de que \textit{average ratings} tem efeito insignificante enquanto \textit{review volume}
    é crítico sugere que quantidade de validação social (volume) supera qualidade numérica (ratings),
    insight relevante para análise de features \texttt{rating} e \texttt{rating\_count} em nosso dataset.

A análise de efeitos de interação (e.g., Discount Rate × Review Volume significativo) demonstra
    importância de considerar relações não-lineares entre categorias produto/marketing,
    conceito explorado em nosso trabalho via SHAP interaction values.
O resultado de que descontos são efetivos apenas quando review volume é alto sugere que
    estratégias de marketing (cupons, patrocínio) podem ter impacto moderado por características de produto,
    hipótese testável em nosso contexto via análise de interações SHAP entre
    \texttt{discount\_percentage}, \texttt{coupons} e \texttt{rating\_count}.

\textbf{Diferenças metodológicas:}
Sharma et al. utilizam Sales Rank como proxy de vendas (variável ordinal convertida em 3 bins para classificação),
    enquanto nosso trabalho analisa \texttt{boughtInLastMonth} (volume absoluto de vendas mensais).
Realizam análise de sentimento de texto de reviews via NLP (Overall Polarity, Sentiment Strengths),
    features não disponíveis em nosso dataset estruturado de marketplace, que
    utiliza atributos diretos como \texttt{sponsored}, \texttt{best\_seller\_badge} e \texttt{coupons}
    ao invés de sentimento textual. 

A abordagem metodológica compara três técnicas
    (Regressão, Decision Tree, ANN)
    avaliando feature importance via análise de coeficientes (regressão),
    predictor importance (DT) e
    pesos de rede (ANN),
    sem utilizar SHAP values para interpretabilidade unificada.
Nosso trabalho padroniza análise de importância via SHAP em modelos tree-based exclusivamente
    (XGBoost, Random Forest, LightGBM),
    fornecendo medidas consistentes e comparáveis via Shapley values
    ao invés de métricas heterogêneas por modelo.

Dataset representa snapshot de dezembro/2017 com Sales Rank coletado em janeiro/2018 (lag temporal de 1 mês),
    enquanto utilizamos dados transversal de setembro/2025 sem séries temporais.
Foco em categoria única (livros) difere de nosso dataset multi-categoria Amazon,
    permitindo análise de generalização cross-category via feature \texttt{category\_id}.

Por fim, embora demonstrem importância de review volume via três modelos,
    não quantificam contribuição relativa agregada produto vs. marketing como categorias conceituais,
    lacuna que nosso trabalho preenche via SHAP values agregados por grupo de features.

\textbf{Li (2022) - A Feature Engineering Approach for Tree-based Machine 
Learning Sales Forecast}

\cite{li2022} propõe um framework sistemático de feature engineering (8C) para
    previsão de vendas em varejo, categorizando features por tipo de impacto:
    \textit{current effect}
    (fatores de impacto direto),
    \textit{carryover effect} (influência de períodos anteriores),
    \textit{collaboration effect} (estratégias combinadas),
    \textit{cost effect} (custos de marketing), 
    \textit{connatural effect} (características do produto),
    \textit{competition effect} (fatores competitivos),
    \textit{customer effect} (preferências do cliente) e
    \textit{correlative effect} (ambiente externo).
Utilizando o dataset Rossmann Store (310.527 amostras, 1.941 dias), os autores aplicaram 
    algoritmos genéticos para gerar 37 novas features, expandindo de 16 para 80 features totais.
Quatro modelos tree-based foram avaliados: XGBoost, LightGBM, CatBoost e Random Forest.

\textbf{Principais resultados:}
Com feature engineering,
    Random Forest alcançou MAPE de 10,2\% e RMSPE de 14,4\%,
    representando melhorias de até 50\% sobre features brutas.
O CatBoost apresentou a maior melhoria relativa (17\% em MAPE).
A análise de feature importance revelou que
    features geradas por combinação de vendas históricas com medianas
    dominaram o ranking de importância em ambos LightGBM e Random Forest,
    demonstrando a eficácia da engenharia de features.

\textbf{Relevância para este trabalho:}
O framework 8C fornece uma estrutura conceitual que pode ser mapeada para nossa dicotomia produto vs. marketing:
    características intrínsecas do produto correspondem ao
    \textit{connatural effect} e \textit{customer effect} (e.g., avaliações, ratings),
    enquanto estratégias de marketing correspondem ao
    \textit{collaboration effect} e \textit{cost effect} (e.g., patrocínio, promoções, cupons).
A ênfase em feature importance analysis alinha-se diretamente com nossa abordagem de 
    interpretabilidade via SHAP.

\textbf{Diferenças metodológicas:}
Li utiliza abordagem temporal (séries de 1.941 dias) focada em previsão de vendas futuras,
    enquanto nosso trabalho adota análise transversal de um snapshot temporal com foco em 
    identificar correlações entre características de produto/marketing e volume de vendas observado.
Além disso, o dataset Rossmann representa varejo físico e online agregado por loja,
    enquanto utilizamos dados de marketplace digital (Amazon) com features específicas de e-commerce como
    patrocínio digital e badges de plataforma.
Por fim, embora Li categorize features conceitualmente,
    não quantifica explicitamente a importância relativa produto vs. marketing, lacuna que nosso trabalho busca preencher.

\textbf{Zhang (2025) - Forecasting E-Market Sales Performance: Use of Big Data Analytics}

\cite{ZHANG2025} investigam fatores determinantes de sales performance em e-marketplaces,
    analisando a moderação de
    \textit{answered questions},
    \textit{review usefulness},
    \textit{product category} e
    \textit{promotional offers}
    sobre a relação entre online reviews e vendas.
Utilizando regressão múltipla com análise de interações, os autores categorizam features em dois grupos:
    características de produto (\textit{online reviews}, \textit{review usefulness}, \textit{product category}) e
    estratégias de marketing (\textit{promotional offers}, \textit{answered questions}).

\textbf{Principais resultados:}
A análise de moderação revelou que \textit{promotional offers} e \textit{product category} 
    possuem papel significativo como moderadores da relação entre reviews e vendas, indicando que
    o efeito de características de produto varia conforme estratégias de marketing aplicadas.
\textit{Review usefulness} demonstrou impacto positivo independente, enquanto
    \textit{answered questions} atuou como moderador fortalecendo a influência de reviews positivos.

\textbf{Relevância para este trabalho:}
Fornece \textbf{framework conceitual} essencial para categorização de features em e-commerce segundo
    dicotomia produto vs. marketing, base teórica diretamente alinhada com nosso problema de pesquisa.
A análise de efeitos de interação (moderação) entre categorias conceituais demonstra importância de considerar
    relações não-lineares, conceito explorado em nosso trabalho via SHAP interaction values.

\textbf{Diferenças metodológicas:}
Utiliza abordagem estatística tradicional (regressão múltipla com termos de interação)
    sem técnicas de interpretabilidade ML (SHAP/feature importance tree-based).
Nosso trabalho expande essa análise categórica utilizando modelos tree-based (XGBoost, Random Forest, LightGBM) e
    quantifica importância relativa produto vs. marketing via SHAP values, fornecendo
    medidas precisas de contribuição marginal ao invés de coeficientes de moderação.
Adicionalmente, o estudo não especifica dataset ou contexto de marketplace,
    enquanto utilizamos dados reais de Amazon com features específicas de plataforma digital.

\subsection{Posicionamento do Presente Trabalho}

\subsection{Posicionamento do Presente Trabalho}

A análise dos trabalhos relacionados revela uma lacuna importante na literatura:
    embora existam estudos sobre previsão de vendas em e-commerce utilizando machine learning,
    há escassez de pesquisas que 
    \textbf{(1)} quantifiquem explicitamente a importância relativa de características de produto versus estratégias de marketing,
    \textbf{(2)} utilizem técnicas modernas de interpretabilidade (SHAP values) para análise unificada entre modelos tree-based, e
    \textbf{(3)} apliquem abordagem transversal em dados de marketplace digital real com features específicas de plataforma.

A Tabela~\ref{tab:related-work-comparison} apresenta uma síntese comparativa dos trabalhos analisados,
    destacando as principais diferenças metodológicas e de escopo em relação ao presente estudo.

\begin{table}[htbp]
\centering
\caption{Comparação entre trabalhos relacionados e o presente estudo}
\label{tab:related-work-comparison}
\small
\begin{tabular}{@{}lcccc@{}}
    \toprule
    \textbf{Aspecto} & \textbf{Sharma} & \textbf{Li} & \textbf{Zhang} & \textbf{Este trabalho} \\
    & \textbf{(2019)} & \textbf{(2022)} & \textbf{(2025)} & \\
    \midrule
    Contexto & Amazon India & Rossmann & E-marketplace & Amazon global \\
    & (livros) & (físico+online) & (genérico) & (multi-categoria) \\
    \addlinespace
    Abordagem & transversal & Temporal & Não especif. & transversal \\
    \addlinespace
    Modelos ML & Reg., DT, ANN & XGB, LGBM, & Regressão & XGB, RF, \\
    & & CatBoost, RF & múltipla & LGBM \\
    \addlinespace
    Interpretab. & Feature imp. & Feature imp. & Coeficientes & \textbf{SHAP values} \\
    & heterogênea & nativa & regressão & \textbf{(unificado)} \\
    \addlinespace
    Análise de & 6 termos de & Não & Moderação & SHAP \\
    interações & interação & reportado & (regressão) & interaction \\
    \addlinespace
    Quantif. & Não & Não & Não & \textbf{Sim} \\
    produto vs. & (por feature) & (framework & (moderação) & \textbf{(agregado)} \\
    marketing & & conceitual) & & \\
    \bottomrule
\end{tabular}
\end{table}

\textbf{Contribuições diferenciadoras deste trabalho:}

\begin{enumerate}
    \item \textbf{Quantificação explícita produto vs. marketing:}
    Enquanto Sharma et al.~(2019) e Zhang~(2025) categorizam features conceitualmente,
        nosso trabalho quantifica empiricamente a contribuição relativa agregada de cada categoria
        via SHAP values somados por grupo de features.
    Esta abordagem fornece uma medida objetiva (em escala absoluta de impacto em vendas)
        da importância de investir em características intrínsecas do produto
        versus estratégias de marketing digital.
    
    \item \textbf{Interpretabilidade unificada via SHAP:}
    Li~(2022) utiliza feature importance nativa de LightGBM/RF,
        enquanto Sharma et al.~(2019) comparam métricas heterogêneas entre modelos
        (coeficientes de regressão, predictor importance de DT, pesos de ANN).
    Nosso trabalho padroniza a análise de importância via SHAP values,
        garantindo comparabilidade rigorosa entre XGBoost, Random Forest e LightGBM
        através de uma métrica teoricamente fundamentada (Shapley values da teoria dos jogos).
    
    \item \textbf{Features específicas de marketplace digital:}
    Diferentemente de Li~(2022), que analisa varejo físico/online tradicional,
        e Zhang~(2025), que utiliza features genéricas (\textit{promotional offers}),
        nosso dataset captura estratégias específicas de plataformas digitais modernas:
        \texttt{sponsored} (patrocínio de visibilidade),
        \texttt{best\_seller\_badge} (gamificação de reputação),
        \texttt{coupons} (promoções direcionadas) e
        \texttt{discount\_percentage} (transparência de desconto).
    Esta granularidade permite análise mais precisa do impacto de táticas de marketing digital
        vs. atributos de produto em ambientes de e-commerce contemporâneos.
    
    \item \textbf{Abordagem transversal para análise de correlações:}
    Enquanto Li~(2022) foca em previsão temporal (forecast de vendas futuras usando séries históricas),
        nosso trabalho adota análise transversal para identificar
        \textbf{quais características presentes no momento atual}
        correlacionam-se com alto volume de vendas.
    Esta distinção é metodologicamente relevante:
        abordagens temporais priorizam padrões de sequência (carryover effects, sazonalidade),
        enquanto nossa análise identifica fatores de diferenciação entre produtos
        em um mesmo período temporal, mais adequado para decisões estratégicas de
        posicionamento de produto e alocação de budget de marketing.
    
    \item \textbf{Contexto empírico Amazon global multi-categoria:}
    Sharma et al.~(2019) limitam-se a livros no marketplace Amazon India,
        enquanto nosso dataset representa múltiplas categorias de produto
        (\texttt{category\_id} como feature)
        no marketplace Amazon global (setembro/2025).
    Esta diversidade permite generalizar achados além de uma categoria específica,
        validando se a importância relativa produto vs. marketing
        varia substancialmente entre tipos de produto.
\end{enumerate}

\textbf{Síntese do posicionamento:}
Este trabalho posiciona-se como uma \textbf{análise interpretativa transversal}
    de importância de features em vendas de e-commerce,
    combinando rigor metodológico de interpretabilidade ML (SHAP)
    com framework conceitual produto vs. marketing
    aplicado a dados reais de marketplace digital contemporâneo.
Diferentemente de estudos focados em previsão temporal (Li)
    ou análise estatística tradicional (Zhang),
    nosso objetivo é \textbf{quantificar e interpretar}
    quais categorias de features (produto ou marketing)
    contribuem mais para vendas observadas em produtos de marketplace,
    fornecendo insights acionáveis para estratégias de produto e marketing digital.

