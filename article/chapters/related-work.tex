\section{Trabalhos Relacionados}
\label{sec:related-work}

Para contextualizar este trabalho, conduzimos uma busca sistemática simplificada na literatura recente sobre
    previsão de vendas em e-commerce utilizando Aprendizado de Máquina.

\subsection{Metodologia de Busca}

A seleção de trabalhos relacionados seguiu a seguinte estratégia:

\textbf{Query de busca:}\\
\texttt{("sales~prediction"~OR~"sales~forecasting")~AND}\\
\texttt{("e-commerce"~OR~"online~retail")~AND}\\
\texttt{("machine~learning"~OR~"regression")}

\textbf{Bibliotecas digitais consultadas:}
\begin{itemize}
    \item IEEE Xplore (\url{https://ieeexplore.ieee.org/})
    \item ACM Digital Library (\url{https://dl.acm.org/})
    \item Google Scholar (\url{https://scholar.google.com/})
\end{itemize}

\textbf{Critérios de inclusão:}
\begin{itemize}
    \item Publicações entre 2020-2025
    \item Abordagem de previsão de vendas em contexto de e-commerce
    \item Uso de técnicas de Machine Learning
    \item Texto completo disponível
\end{itemize}

\textbf{Estratégia de seleção:} 
Para cada biblioteca, os resultados foram ordenados por relevância e os dois 
primeiros artigos que atenderam aos critérios de inclusão foram selecionados, 
totalizando 6 trabalhos analisados.

\subsection{Análise dos Trabalhos}

% Para cada trabalho, descrever:
% 1. Objetivo/problema abordado
% 2. Metodologia/algoritmos utilizados
% 3. Principais resultados
% 4. Limitações ou lacunas

\textbf{[Autor et al., Ano] - Título do Trabalho 1 (IEEE)}

[Resumo do trabalho: problema, metodologia, resultados principais]

\textbf{[Autor et al., Ano] - Título do Trabalho 2 (IEEE)}

[...]

% Continuar para os 6 trabalhos

\subsection{Posicionamento do Presente Trabalho}

Embora os trabalhos analisados demonstrem a aplicabilidade de ML em previsão 
de vendas, identificamos as seguintes lacunas/oportunidades:

\begin{itemize}
    \item \textbf{Interpretabilidade:} Poucos trabalhos focam em explicar 
          quais features são mais importantes [se aplicável]
    
    \item \textbf{Comparação produto vs. marketing:} Não identificamos 
          estudos que quantificam sistematicamente o impacto relativo dessas 
          categorias de fatores
    
    \item \textbf{Dataset contemporâneo:} Utilizamos dados de 2025 do 
          marketplace Amazon, capturando features modernas (sponsored products, 
          buy box, etc.)
    
    \item \textbf{Reprodutibilidade:} Disponibilizamos código e dados 
          públicos, facilitando validação e extensão
\end{itemize}

Este trabalho contribui ao aplicar técnicas de interpretabilidade (SHAP, 
feature importance) para comparar sistematicamente o impacto de características 
intrínsecas e estratégias de marketing em um dataset de larga escala.
