\section{Trabalhos Relacionados}
\label{sec:related-work}

Para contextualizar este trabalho, conduzimos uma busca sistemática simplificada na literatura recente sobre
    previsão de vendas em e-commerce utilizando Aprendizado de Máquina.

\subsection{Metodologia de Busca}

A seleção de trabalhos relacionados seguiu a seguinte estratégia:

\textbf{Query de busca:}\\
\texttt{("sales~prediction"~OR "sales~forecasting")~AND}\\
\texttt{("e-commerce"~OR "online~retail")~AND}\\
\texttt{("machine~learning"~OR "regression")~AND}\\
\texttt{("feature~importance"~OR "product~characteristics"~OR "marketing~strategies")}

\textbf{Bibliotecas digitais consultadas:}
\begin{itemize}
    \item IEEE Xplore (\url{https://ieeexplore.ieee.org/})
    \item ScienceDirect (\url{https://www.sciencedirect.com/})
    \item SpringerLink (\url{https://link.springer.com/})
\end{itemize}

\textbf{Critérios de inclusão:}
\begin{itemize}
    \item Publicações entre 2020-2025
    \item Abordagem de previsão de vendas em contexto de e-commerce ou 
          varejo online
    \item Uso de técnicas de Machine Learning para regressão
    \item Análise de features relacionadas a produto e/ou marketing
    \item Texto completo disponível via acesso institucional
\end{itemize}

\textbf{Critérios de exclusão:}
\begin{itemize}
    \item Trabalhos focados exclusivamente em séries temporais sem 
          análise de features de produto ou marketing
    \item Trabalhos puramente metodológicos sem aplicação em e-commerce
    \item Trabalhos de classificação (e.g., prever categoria best-seller) 
          em vez de regressão (prever valor de vendas)
\end{itemize}

\textbf{Estratégia de seleção:} Para cada biblioteca digital, os resultados 
foram ordenados por relevância. Priorizando os critérios acima, os dois 
primeiros artigos que atenderam aos critérios de inclusão foram selecionados, 
totalizando 6 trabalhos analisados. Para artigos com títulos e abstracts 
promissores mas com alinhamento incerto, realizou-se leitura da introdução 
e análise de figuras/tabelas para validação antes da inclusão final.

\subsection{Análise dos Trabalhos}

% Para cada trabalho, descrever:
% 1. Objetivo/problema abordado
% 2. Metodologia/algoritmos utilizados
% 3. Principais resultados
% 4. Limitações ou lacunas

\textbf{Li (2022) - A Feature Engineering Approach for Tree-based Machine 
Learning Sales Forecast}

\cite{li2022} propõe um framework sistemático de feature engineering (8C) 
para previsão de vendas em varejo, categorizando features por tipo de 
impacto: \textit{current effect} (fatores de impacto direto), \textit{carryover 
effect} (influência de períodos anteriores), \textit{collaboration effect} 
(estratégias combinadas), \textit{cost effect} (custos de marketing), 
\textit{connatural effect} (características do produto), \textit{competition 
effect} (fatores competitivos), \textit{customer effect} (preferências do 
cliente) e \textit{correlative effect} (ambiente externo). Utilizando o 
dataset Rossmann Store (310.527 amostras, 1.941 dias), os autores aplicaram 
algoritmos genéticos para gerar 37 novas features, expandindo de 16 para 80 
features totais. Quatro modelos tree-based foram avaliados: XGBoost, LightGBM, 
CatBoost e Random Forest.

\textbf{Principais resultados:} Com feature engineering, Random Forest 
alcançou MAPE de 10,2\% e RMSPE de 14,4\%, representando melhorias de até 
50\% sobre features brutas. O CatBoost apresentou a maior melhoria relativa 
(17\% em MAPE). A análise de feature importance revelou que features geradas 
por combinação de vendas históricas com medianas dominaram o ranking de 
importância em ambos LightGBM e Random Forest, demonstrando a eficácia da 
engenharia de features.

\textbf{Relevância para este trabalho:} O framework 8C fornece uma estrutura 
conceitual que pode ser mapeada para nossa dicotomia produto vs. marketing: 
características intrínsecas do produto correspondem ao \textit{connatural 
effect} e \textit{customer effect} (e.g., avaliações, ratings), enquanto 
estratégias de marketing correspondem ao \textit{collaboration effect} e 
\textit{cost effect} (e.g., patrocínio, promoções, cupons). A ênfase em 
feature importance analysis alinha-se diretamente com nossa abordagem de 
interpretabilidade via SHAP.

\textbf{Limitações e lacunas:} Embora Li (2022) categorize features por tipo 
de impacto, o trabalho não quantifica explicitamente a importância relativa 
de cada categoria (produto vs. marketing vs. temporal). O dataset utilizado 
(Rossmann Store) representa varejo físico e online agregado por loja, carecendo 
de features específicas de e-commerce puro como patrocínio digital, cupons 
online e badges de plataforma (e.g., best seller). Além disso, o foco em 
\textit{carryover effect} (séries temporais) difere de nossa abordagem 
cross-sectional que prioriza características estáticas de produto versus 
dinâmicas de marketing.



\subsection{Posicionamento do Presente Trabalho}

% Apresentar tabela comparativa dos trabalhos relacionados
% Destacar como o presente trabalho se diferencia

