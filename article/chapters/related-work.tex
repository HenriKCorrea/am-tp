\section{Trabalhos Relacionados}
\label{sec:related-work}

A literatura sobre previsão de vendas em e-commerce utilizando Machine Learning é extensa,
    mas há escassez de estudos que
    \textbf{(1)} quantifiquem explicitamente a importância relativa de
        produto versus estratégias de marketing e
    \textbf{(2)} apliquem abordagem transversal em dados de marketplace digital.
Esta seção analisa trabalhos relevantes segundo essas dimensões.

\subsection{Análise de Importância Produto vs. Marketing}

\cite{sharma2019} investigam vendas de livros na Amazon India (1.408 produtos)
    comparando regressão múltipla, Decision Tree (C5.0) e Redes Neurais (ANN).
Analisam features de produto (\textit{price}, \textit{average ratings}, \textit{review volume})
    e marketing (\textit{discount rate}, \textit{free delivery}),
    extraindo sentimentos de reviews via NLP.
\textit{review volume} (produto) domina importância em todos os modelos
    (ANN: 0.17), enquanto \textit{discount rate} (marketing) tem impacto mínimo (0.01-0.06),
    contradizendo teoria de transaction utility.
Interações como Discount Rate x Review Volume foram significativas,
    sugerindo que marketing é efetivo apenas quando características de produto (volume de reviews) são fortes.
Utilizam feature importance heterogênea entre modelos
    (coeficientes de regressão, predictor importance de DT, pesos de ANN)
    sem interpretabilidade unificada via SHAP.
Dataset limitado a livros (categoria única) e
    não quantificam contribuição relativa agregada produto vs. marketing
    como categorias conceituais.

\cite{zhao2020} analisam fatores de vendas em Taobao
    categorizando features em quatro grupos:
    \textit{online reviews},
    \textit{review system curation} (clustering de sentimentos),
    \textit{promotional marketing} (discount rate, cupons) e
    \textit{seller guarantees} (7-day returns, freight insurance).
Utilizando redes neurais (SFNN model) com matriz de pesos,
    identificam que \textit{número de palavras de sentimento} (review curation) tem maior impacto,
    seguido por \textit{review volume} e \textit{discount rate}.
Fornecem framework conceitual para categorização produto vs. marketing,
    mas empregam pesos de rede neural para interpretação
    ao invés de técnicas agnósticas a modelos como SHAP.

\subsection{Feature Engineering e Interpretabilidade}

\cite{li2022} propõe framework 8C de feature engineering para forecast de vendas,
    categorizando features por tipo de impacto:
    \textit{connatural effect} (características de produto),
    \textit{cost effect} (custos de marketing),
    \textit{carryover effect} (influência temporal), entre outros.
Utilizando dataset Rossmann (310.527 amostras, 1.941 dias),
    aplicam algoritmos genéticos para gerar 37 novas features,
    alcançando MAPE de 10.2\% (Random Forest).
Feature importance nativa de LightGBM/RF revelou domínio de features temporais
    (combinação de vendas históricas com medianas).
Utiliza abordagem temporal (séries de 1.941 dias) focada em forecast,
    enquanto nosso trabalho adota análise transversal para identificar correlações.
Dataset de varejo físico/online agregado por loja,
    sem features específicas de marketplace digital (patrocínio, badges, cupons).
Embora categorize features conceitualmente,
    não quantifica importância relativa produto vs. marketing via agregação.

\cite{diwandari2025} exploram métodos de seleção de features
    (Information Gain, Chi-Square, Fisher's Discriminant Ratio)
    para predição de compras em e-commerce usando clickstream data.
Categorizam features em User, Item, Category, Customer-Item e Customer-Category,
    identificando que \textit{probabilidade histórica de compra} e \textit{atributos temporais}
    (day of month, day of week) apresentam maior poder discriminativo.
Chi-Square minimizou class overlap entre compradores/não-compradores.
Demonstram importância de categorização sistemática de features,
    mas focam em \textbf{seleção pré-modelo} (feature engineering)
    ao invés de \textbf{interpretabilidade pós-modelo} via SHAP.

\subsection{Posicionamento do Presente Trabalho}

A Tabela~\ref{tab:related-work-comparison} sintetiza as diferenças metodológicas entre os trabalhos analisados.

\begin{table}[htbp]
\centering
\caption{Comparação entre trabalhos relacionados e o presente estudo}
\label{tab:related-work-comparison}
\small
\begin{tabular}{@{}lcccc@{}}
    \toprule
    \textbf{Aspecto} & \textbf{Sharma} & \textbf{Li} & \textbf{Zhao} & \textbf{Este trabalho} \\
    & \textbf{2019} & \textbf{2022} & \textbf{2020} & \\
    \midrule
    Contexto & Amazon India & Rossmann & Taobao & Amazon global \\
    & (livros) & (físico+online) & (e-comm.) & (eletrônicos) \\
    \addlinespace
    Abordagem & Transversal & Temporal & Transversal & Transversal \\
    & (snapshot) & (1.941 dias) & (snapshot) & (snapshot) \\
    \addlinespace
    Modelos ML & Reg., DT, ANN & XGB, LGBM, & SFNN (NN) & XGB, RF, \\
    & & CatBoost, RF & & LGBM \\
    \addlinespace
    Interpretab. & Feature imp. & Feature imp. & Pesos de rede & \textbf{SHAP values} \\
    & heterogênea & nativa & neural & \textbf{(unificado)} \\
    \addlinespace
    Quantif. & Não & Não & Não & \textbf{Sim} \\
    produto vs. & (por feature) & (framework & (4 categorias) & \textbf{(agregado)} \\
    marketing & & conceitual) & & \\
    \bottomrule
\end{tabular}
\end{table}

\textbf{Contribuições diferenciadoras:}

\begin{enumerate}
    \item \textbf{Quantificação explícita produto vs. marketing:}
    Enquanto \cite{sharma2019} e \cite{zhao2020} analisam features individualmente,
        nosso trabalho agrega SHAP values por categoria conceitual,
        fornecendo medida objetiva (em escala percentual) da importância relativa
        de características de produto versus estratégias de marketing.
    
    \item \textbf{Interpretabilidade unificada via SHAP:}
    Diferentemente de \cite{sharma2019}, que compara métricas heterogêneas entre modelos,
        e \cite{li2022}, que utiliza feature importance nativa,
        padronizamos análise via SHAP values (Shapley values da teoria dos jogos),
        garantindo comparabilidade rigorosa entre XGBoost, Random Forest e LightGBM.
    
    \item \textbf{Features específicas de marketplace digital:}
    Nosso dataset captura estratégias de plataformas modernas
        (\texttt{is\_sponsored}, \texttt{has\_coupon}, \texttt{is\_best\_seller}, \texttt{sustainability\_tags}),
        permitindo análise mais precisa do impacto de táticas de marketing digital
        versus atributos de produto em e-commerce contemporâneo.
    
    \item \textbf{Abordagem transversal para análise de correlações:}
    Enquanto \cite{li2022} foca em forecast temporal (padrões de sequência, sazonalidade),
        identificamos \textbf{quais características presentes no momento atual}
        correlacionam-se com alto volume de vendas.
\end{enumerate}

