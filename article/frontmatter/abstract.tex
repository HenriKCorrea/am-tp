\begin{resumo} 

A previsão precisa de vendas é fundamental para a gestão eficiente de 
    e-commerce, impactando diretamente o planejamento de estoque, logística e estratégias comerciais.
Embora seja conhecido que tanto características do produto quanto estratégias de marketing influenciam vendas,
    a proporção desses efeitos e como eles se comportam em conjunto em um marketplace de larga escala permanece
    uma questão empírica relevante.
Este trabalho apresenta uma abordagem de Aprendizado de Máquina para prever 
    vendas mensais em e-commerce e investigar o impacto relativo dessas
    duas categorias de fatores.
Utilizamos o Amazon Products Sales Dataset, contendo
    42.675 produtos e
    17 atributos, incluindo
    variáveis numéricas, categóricas e
    temporais.
Após análise exploratória e pré-processamento 
criterioso dos dados, avaliamos o desempenho dos algoritmos de regressão
    [INSERIR ALGORITMOS].
A estratégia de avaliação empregou
    [INSERIR METODOLOGIA] % validação cruzada k-fold
    e métricas como
    [INSERIR MÉTRICAS]. % RMSE, MAE e R²
Os resultados demonstram que
    [INSERIR MELHOR ALGORITMO]. % [ALGORITMO] alcançou o melhor desempenho, com R² de [VALOR] no conjunto de teste.
A análise de interpretabilidade revelou que
    [INSERIR MELHOR ANÁLISE]. % estratégias de marketing, especialmente [FEATURE PRINCIPAL], apresentam maior poder preditivo que características intrínsecas do produto, respondendo por aproximadamente [X]\% da importância total das features.
Tais resultados sugerem que
    [INSERIR HIGHLIGHT CONCLUSÃO]. % investimentos em visibilidade e promoções têm impacto mais imediato nas vendas do que melhorias na qualidade do produto a curto prazo.

\vspace{0.5em}
\noindent\textbf{Palavras-chave:}
    Aprendizado~de~Máquina,
    Previsão~de~Vendas,
    E-commerce,
    Interpretabilidade.

\end{resumo}

\begin{abstract}

Accurate sales forecasting is fundamental for efficient e-commerce management,
    directly impacting inventory planning, logistics, and business strategies.
Although it is known that both product characteristics and marketing strategies 
    influence sales, the magnitude of these effects and how they behave together 
    in a large-scale marketplace remains a relevant empirical question.
This work presents a Machine Learning approach to predict monthly e-commerce 
    sales and investigate the relative impact of these two categories of factors.
We use the Amazon Products Sales Dataset, containing 42,675 products and 
    17 attributes, including numerical, categorical, and temporal variables.
After exploratory analysis and careful data preprocessing, we evaluated the performance of regression algorithms
    [INSERT ALGORITHMS].
The evaluation strategy employed
    [INSERT METHODOLOGY]
    and metrics such as
    [INSERT METRICS].
The results demonstrate that
    [INSERT BEST ALGORITHM].
These results suggest that
    [INSERT HIGHLIGHT CONCLUSION].

\vspace{0.5em}
\noindent\textbf{Keywords:}
    Machine~Learning,
    Sales~Forecasting,
    E-commerce,
    Interpretability.

\end{abstract}
